\chapter{Propulsion Analysis} \label{cp:propulsion}

% Lucas

% Briefly introduce the topic/purpose of this chapter, what information can be found in this chapter, and the key findings.

In this chapter of the report, we will analyze electrical power draw (W) plotted against thrust (Oz) for our motor-propeller configuration tested in the wind tunnel. Data were collected for different airspeeds: static (0 MPH), aircraft stall (22.8 MPH), 30 MPH, aircraft cruise speed (35 MPH), and 40 MPH.

In the presented plots, we also compare the wind tunnel collected data with MotoCalc simulation outputs. However, the MotoCalc data are for a configuration of an E-Flite Power 32 motor with an APC 13x6.5E propeller. When using an APC 13x4.5E propeller (as in the wind tunnel test) for the simulation, the output thrust at all airspeeds was 0.0 for all throttle configurations. We decided to use data from a MotoCalc simulation of the same motor with an APC 13x6.5E propeller, considering its output close matched to the collected data from the wind tunnel testing.

% Include plots here

We can observe from both graphs that our motor have more than enough thrust output to propel our aircraft. Our design thrust required is XX Oz at cruise speed (35 MPH). To maintain steady level flight (thrust available = thrust required) at 35 MPH the electrical power draw is XX W. With this, our nominal cruise speed flight time is XX:XX.

% Compare real data to MotoCalc

% Lessons learned
